\documentclass[12pt,]{article}
\usepackage{lmodern}
\usepackage{setspace}
\setstretch{1.5}
\usepackage{amssymb,amsmath}
\usepackage{ifxetex,ifluatex}
\usepackage{fixltx2e} % provides \textsubscript
\ifnum 0\ifxetex 1\fi\ifluatex 1\fi=0 % if pdftex
  \usepackage[T1]{fontenc}
  \usepackage[utf8]{inputenc}
\else % if luatex or xelatex
  \ifxetex
    \usepackage{mathspec}
  \else
    \usepackage{fontspec}
  \fi
  \defaultfontfeatures{Ligatures=TeX,Scale=MatchLowercase}
\fi
% use upquote if available, for straight quotes in verbatim environments
\IfFileExists{upquote.sty}{\usepackage{upquote}}{}
% use microtype if available
\IfFileExists{microtype.sty}{%
\usepackage{microtype}
\UseMicrotypeSet[protrusion]{basicmath} % disable protrusion for tt fonts
}{}
\usepackage[margin=1in]{geometry}
\usepackage{hyperref}
\PassOptionsToPackage{usenames,dvipsnames}{color} % color is loaded by hyperref
\hypersetup{unicode=true,
            colorlinks=true,
            linkcolor=blue,
            citecolor=blue,
            urlcolor=blue,
            breaklinks=true}
\urlstyle{same}  % don't use monospace font for urls
\usepackage{longtable,booktabs}
\usepackage{graphicx,grffile}
\makeatletter
\def\maxwidth{\ifdim\Gin@nat@width>\linewidth\linewidth\else\Gin@nat@width\fi}
\def\maxheight{\ifdim\Gin@nat@height>\textheight\textheight\else\Gin@nat@height\fi}
\makeatother
% Scale images if necessary, so that they will not overflow the page
% margins by default, and it is still possible to overwrite the defaults
% using explicit options in \includegraphics[width, height, ...]{}
\setkeys{Gin}{width=\maxwidth,height=\maxheight,keepaspectratio}
\IfFileExists{parskip.sty}{%
\usepackage{parskip}
}{% else
\setlength{\parindent}{0pt}
\setlength{\parskip}{6pt plus 2pt minus 1pt}
}
\setlength{\emergencystretch}{3em}  % prevent overfull lines
\providecommand{\tightlist}{%
  \setlength{\itemsep}{0pt}\setlength{\parskip}{0pt}}
\setcounter{secnumdepth}{5}
% Redefines (sub)paragraphs to behave more like sections
\ifx\paragraph\undefined\else
\let\oldparagraph\paragraph
\renewcommand{\paragraph}[1]{\oldparagraph{#1}\mbox{}}
\fi
\ifx\subparagraph\undefined\else
\let\oldsubparagraph\subparagraph
\renewcommand{\subparagraph}[1]{\oldsubparagraph{#1}\mbox{}}
\fi

%%% Use protect on footnotes to avoid problems with footnotes in titles
\let\rmarkdownfootnote\footnote%
\def\footnote{\protect\rmarkdownfootnote}

%%% Change title format to be more compact
\usepackage{titling}

% Create subtitle command for use in maketitle
\newcommand{\subtitle}[1]{
  \posttitle{
    \begin{center}\large#1\end{center}
    }
}

\setlength{\droptitle}{-2em}

  \title{}
    \pretitle{\vspace{\droptitle}}
  \posttitle{}
    \author{}
    \preauthor{}\postauthor{}
    \date{}
    \predate{}\postdate{}
  
%%%%%%%%%%
%% This part of the preamble was copied from dg_template.text
%% Load De Gruyter specific settings 
\usepackage{dgjournal}          

%% The mathptmx package is recommended for Times compatible math symbols.
%% Use mtpro2 or mathtime instead of mathptmx if you have the commercially
%% available MathTime fonts.
%% Other options are txfonts (free) or belleek (free) or TM-Math (commercial)
\usepackage{mathptmx}

%% Use the graphics package to include figures
\usepackage{graphics}

%% Use natbib with these recommended options
\usepackage[authoryear,comma,longnamesfirst,sectionbib]{natbib} 
%%%%%%%%%%

%% This part of the preamble has other useful packages and new command, which can be edited as desired, and can be used with either the DeGruyter or Arxiv preambles
%\usepackage{amsthm, amsfonts, dsfont}
%\usepackage[colorlinks=true, linkcolor=blue, urlcolor=blue, citecolor=blue]{hyperref}
%\usepackage{booktabs} % for the bold lines above and below the fancy tables
%\usepackage{array}
%\usepackage[dvips]{graphicx}
%\usepackage{multirow}
%\usepackage{caption}
%\usepackage{setspace}

% Table spacing
%\setlength{\tabcolsep}{4pt}

% Figures
%\DeclareGraphicsExtensions{.pdf}
%\graphicspath{{figures/}}


\usepackage{amsthm}
\newtheorem{theorem}{Theorem}[section]
\newtheorem{lemma}{Lemma}[section]
\theoremstyle{definition}
\newtheorem{definition}{Definition}[section]
\newtheorem{corollary}{Corollary}[section]
\newtheorem{proposition}{Proposition}[section]
\theoremstyle{definition}
\newtheorem{example}{Example}[section]
\theoremstyle{definition}
\newtheorem{exercise}{Exercise}[section]
\theoremstyle{remark}
\newtheorem*{remark}{Remark}
\newtheorem*{solution}{Solution}
\begin{document}

\hypertarget{this-should-be-the-first-section-of-your-document}{%
\section{This should be the first section of your
document}\label{this-should-be-the-first-section-of-your-document}}

** \textbf{Note: Please see the YAML header of this file for information
about front matter and for an easy way to reformat your article for
submitting to Arxiv. For further details, see
\texttt{?degruyter\_article}.} **

Your text goes here. Start with the first section or paragraph of your
article. Do not set use either headers or footers and do not set any
running heads or change any page numbers.

The title page along with headers and footers will be inserted by the
EdiKit system. Use the ``revise manuscript'' link to enter this
information in the EdiKit system.

Use the standard LaTeX commands to set the text of the article. The
dgjournal package works with any of the standard LaTeX document classes
- article, report or book. Other document classes should work but
compliance with De Gruyter requirements is not assured. For more
information on LaTeX, see (Lamport
\protect\hyperlink{ref-lamport}{1994}; Mittelbach et al.
\protect\hyperlink{ref-mittelbach}{2004}; Oetiker
\protect\hyperlink{ref-oetiker}{2008}). (Mittelbach et al.
\protect\hyperlink{ref-mittelbach}{2004}) is highly recommended.

Use the standard LaTeX sectioning commands for your headings.

\hypertarget{a-second-order-heading}{%
\subsection{A second order heading}\label{a-second-order-heading}}

Some text under the subheading. Paragraphs that follow heads are not
indented.

Math should also be set in Times. Use the mathptmx package if you do not
have any of the commercially available fonts that are compatible with
Times.

\begin{equation}
    y^{(n)} = \sum_{i=0}^{n-1} a_i(x) y^{(i)} + r(x) 
\end{equation}

All environments provided by the standard LaTeX document classes are
unchanged. Vertical spaces within lists have been altered to comply with
De Gruyter requirements.

\begin{enumerate}
\item This is the first item within the list. Some more text here in order to
  display the alignment.
\item Another item in the list.
\item Yet another item in the list.
\end{enumerate}

Here is an example of a Figure. It's the same as in standard LaTeX.

\begin{figure}[!h]
%% Use the graphics package to insert figures
%% \includegraphics{figure.eps}
% Use \centering to center the table
\centering
%% A small box in place of a figure
\framebox{%
  \begin{minipage}{10pc}
    \begin{center}
      \vspace{1cm}\par
      A figure\par
      \vspace{1cm}
    \end{center}
  \end{minipage}}
\caption{Insert your caption here. If you wish to label your figure for
  cross-referencing, use a label either within the caption or after it.}
\label{fig1}
\end{figure}

An example of a table follows. This is also the same as in standard
LaTeX.

\begin{table}[!h]
% Use \centering to center the table
\centering
\caption{Insert your table caption here. If you wish to label the table for
  cross-referencing, use a label either within the caption or after it.}
\begin{tabular}{llll}
\hline
Symbol        & LaTeX Command      & Symbol      & LaTeX Command \\
\hline
$\alpha$      & \verb+\alpha+      & $\zeta$     & \verb+\zeta+ \\
$\beta$       & \verb+\beta+       & $\eta$      & \verb+\eta+ \\
$\gamma$      & \verb+\gamma+      & $\theta$    & \verb+\theta+ \\
$\delta$      & \verb+\delta+      & $\vartheta$ & \verb+\vartheta+ \\
$\epsilon$    & \verb+\epsilon+    & $\iota$     & \verb+\iota+ \\
$\varepsilon$ & \verb+\varepsilon+ & $\kappa$    & \verb+\kappa+ \\
\hline
\end{tabular}
\end{table}

Use the \verb+thebibliography+ environment for the references. BibTeX
users may use the provided BibTeX style file DeGruyter.bst.

\hypertarget{references}{%
\section{References}\label{references}}

\bibliographystyle{DeGruyter}
\bibliography{bibliography}

\hypertarget{refs}{}
\leavevmode\hypertarget{ref-lamport}{}%
Lamport, Leslie. 1994. \emph{LaTeX: A Document Preparation System:
User's Guide and Reference Manual}. Second. Reading, MA, USA:
Addison-Wesley.

\leavevmode\hypertarget{ref-mittelbach}{}%
Mittelbach, Frank, Michel Goossens, Johannes Braams, David Carlisle,
Chris Rowley, Christine Detig, and Joachim Schrod. 2004. \emph{The LaTeX
Companion}. Second. Tools and Techniques for Computer Typesetting.
Reading, MA, USA: Addison-Wesley.

\leavevmode\hypertarget{ref-oetiker}{}%
Oetiker, Tobias. 2008. \emph{The Not so Short Introduction to}. 4.26 ed.
\url{http://ctan.tug.org/tex-archive/info/lshort/english/lshort.pdf}.


\end{document}
